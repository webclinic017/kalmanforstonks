
\documentclass[12pt, letterpaper]{article}
\usepackage[margin=.5in]{geometry}
\usepackage{url}
\usepackage{graphicx}
\usepackage{caption}
\usepackage{subcaption}

\newcommand{\code}[1]{\texttt{#1}}

\title{Kalman Filtering for Realizing Stock Market Processes}
\author{Omar Riaz \\ Northeastern University}
\date{2021 April}

\begin{document}
\maketitle
\section{Introduction}
Eugene Fama, the Father of the efficient market theory, described the total accumulation of noise onto stock prices as independent, random, and unrelated to real world events, in order to justify a random walk model. The random walk hypothesis has since been disproven by autocorrelation functions and spectral analysis, \cite{rankin}. Market prices have non-stationary means, and correlation between consecutive prices. A mathematical model that harnesses these correlations while considering the uncertainty in noisy stock data, relative to some steady instrinsic market value, would be suited to the application of realizing this Brownian stock market motion \cite{martinelli}. In this project a Kalman filter will be used to estimate a smooth trend-line within the data which represents instrinsic value before perturbation by market noise. \cite{arnold}

Kalman filtering is an algorithm usually performed on time series data with statistical noise, which produces estimates of a an unknown true state.  We assume a time series' process can be modeled recursively:
\[x_{k+1} = \Phi x_k + w_k\]
\[z_k = Hx_k + v_k\]
where $z_k$ is our measurement vector (market price), and $x_k$ is our state vector (intrinsic value). Both $w_k$ and $v_k$ are Gaussian white noise models with covariance matrices called $Q_k$ and $R_k$, respectively. Given  parameters for the process and an a priori Gaussian estimate of the state's mean $\mu_k$ and covariance $P_k'$, a prediction of the initial state $x_k'$ is formed. Then, when the measurement $z_k$ is observed, both this prediction and the state covariance are updated:
\[x_k = x_k' + K_k(z_k - Hx_k')\]
\[P_k  = (I - K_kH)P_k'\]
where $K_k$ is the Kalman gain: a minimum mean-square error estimator. The updated state estimation is then projected into the prediction of $x_{k+1}'$ using the recursive process model. 


\section{Methods}

To define the filter, the following matrices must be specified:
$\Phi$, the state-transition model,
$H$, the observation model,
$Q$, the covariance of the process noise,
$R$, the covariance of the observation noise

Because we assume the market noise to be completely random, the observation model is assumed as unity. The state-transition model correlates to market momentum which is calculated as the previous day's closing price subtracted from the current closing price.

Bassett \cite{bassett} suggests initalizing $Q_k$ to the empircal covariance from the previous day. While using the covariance between two days was be sufficient in estimating the uncertainty in consecutive states, this covariance was also inflated by including varying numbers of prior market prices.  $R_k$ was based on the bid ask spread of 1/8 tick size, as the discrepancy between ask price and bid price indicates tight liquidity and thus uncertainty \cite{labbe}.

The Gaussian distribution representing the initial intrinsic value state $x_0'$ may be determined using a variety of valuation techniques. For the purposes of this project, the initial state $x_-1$ will be chosen as the last closing price. Likewise, fundamental analysis techniques would traditionally be employed to manage risk and estimate an initial uncertainty, but we initialize $P_0' = Q$. 

An Expectation-Maximization algorithm is also implemented to optimize model parameters. However, this optimization problem is non-convex, meaning it may converge off the global maximum, so sensible initial parameters were prepared. The EM algorithm seeks to maximize the likelihood of observation given the current state mean and covariance.

The Kalman filter is implemented in Python using the \code{pykalman} library\cite{duckworth}. A Kalman smoother function is also tested, which iterates through and updates the entire history of states every time a measurement is observed. Because the smoother estimate considers future data, it has no advantage in prediction.

\section{Results}
The result for applying the Kalman filter to the AAPL and GME stock data from the duration of January 1st, 2020 through June 1st, 2020, is shown in the Results Appendix A. Each figure contains a plot of the stock data, the Kalman estimate of the data, and the Kalman smoother estimate. Two figures for each ticker are shown. The first with initial covariance calculated from prior bar data, the second with initial parameters determined with EM.

\bibliographystyle{unsrt}
\bibliography{bibby}

\appendix
\section{Results Appendix}


\begin{figure}[htbp!]
\centering
  \centering
  \includegraphics[width=1\linewidth]{aapl.png}
  \caption{AAPL Kalman Estimation}

  \centering
  \includegraphics[width=1\linewidth]{gme.png}
  \caption{GME Kalman Estimation}
\end{figure}

\begin{figure}[htbp!]
\centering
  \centering
  \includegraphics[width=1\linewidth]{aapl_EM.png}
  \caption{AAPL Kalman Estimation with EM optimized parameters}

  \centering
  \includegraphics[width=1\linewidth]{gme_EM.png}
  \caption{GME Kalman Estimation with EM optimized parameters}
\end{figure}


\end{document}

